\documentclass{article}
\usepackage[top=0.85in, bottom=0.9in, left=0.7in, right=0.7in]{geometry}
\usepackage{mdwlist}
\usepackage{enumitem}
\usepackage{etaremune}
\usepackage{longtable}
\usepackage{verbatim}
\usepackage{fancyhdr}
\usepackage[colorlinks=True, urlcolor=cyan]{hyperref}
\usepackage{helvet}
\renewcommand{\familydefault}{\sfdefault}
\pagestyle{fancy}
\pagenumbering{arabic}
\lhead{\itshape W. Andrew Barr - CV}
\chead{}
\rhead{\thepage}
\lfoot{}
\cfoot{Updated \today}
\rfoot{}

\thispagestyle{empty}

%custom environment to control hanging indent in description
\newenvironment{mylist}
{\begin{description}[style=unboxed,leftmargin=1.3cm]}
{\end{description}}


\begin{document}
\begin{center}
\noindent{\bfseries{\Huge W. Andrew Barr - Curriculum Vitae}}
\end{center}

\vspace{15pt}

\noindent\begin{minipage}{.60\textwidth}
\begin{flushleft}
Center for the Advanced Study of Human Paleobiology\\
Department of Anthropology\\
The George Washington University\\
ORCID: \href{https://orcid.org/0000-0001-9763-6440}{0000-0001-9763-6440}\\
\end{flushleft}
\end{minipage}
\begin{minipage}{.395\textwidth}
\begin{flushright}
800 22nd St NW, Suite 6000\\
Washington, DC 20052 \\
wabarr@email.gwu.edu\\
+1 (202) 994-3213\\
\end{flushright}
\end{minipage}


\noindent\rule[-2mm]{\textwidth}{1pt}

\section*{Education}

\begin{tabular}{p{.1\textwidth}p{0.86\textwidth}}
2014 & University of Texas at Austin. Ph.D., Anthropology. \\[4pt] %\emph{The Paleoenvironments of Early Hominins in the Omo Shungura Formation (Plio-Pleistocene, Ethiopia): Synthesizing Multiple Lines of Evidence Using Phylogenetic Ecomorphology}.
2008 & University of Texas at Austin. M.A., Anthropology. \\[4pt]
2005 & Tulane University. B.S., Anthropology and French.\\
\end{tabular} 


\section*{Academic Appointments}
\begin{tabular}{p{.15\textwidth}p{0.8\textwidth}}
2019 - Present & Assistant Professor. Center for the Advanced Study of Human Paleobiology. Department of Anthropology. The George Washington University.\\[4pt]
2014 - Present & Research Associate. Department of Paleobiology.  National Museum of Natural History.\\
2016 - 2019 & Visiting Assistant Professor. Center for the Advanced Study of Human Paleobiology. Department of Anthropology. The George Washington University.\\[4pt]
2014 - 2016 & Postdoctoral Scientist. Center for the Advanced Study of Human Paleobiology. Department of Anthropology. The George Washington University. Advisor: Bernard Wood.\\[4pt]
\end{tabular} 

\section*{Research Grants and Awards}
\begin{tabular}{p{.1\textwidth}p{0.86\textwidth}}
2021 & National Science Foundation - Examining the relationship between an increasingly carnivorous \emph{Homo erectus} and Pleistocene mammal extinctions. Role: PI. \$90,099\\[4pt]
2020 & National Science Foundation - Collaborative Research: Catching Fire: Pyrotechnology and Ecosystem Change in the Turkana Basin. Role: Co-PI. \$237,661.\\[4pt]
2019 &  National Science Foundation - Collaborative Research: REU Site: Past and Present Human-Environment Dynamics in the Turkana Basin, Kenya. Role: Senior Personnel. \$305,846\\[4pt]
2019 & National Science Foundation - HRRBAA: Paleontology and paleoanthropology of a potential Late Miocene site in the Laikipia highlands. Role: PI. \$26,581. 2020 and 2021 fieldwork seasons postponed due to COVID-19. \\[4pt]
2018 & American Association of Physical Anthropologists - Professional Development Award. \$7,500\\[4pt]

\end{tabular}

\section*{Peer-Reviews Publications Most Relevant to This Proposal}

\begin{itemize}[label={},leftmargin=*]
%% For publications, include Citation, with period afterward. Then include DOI if available, with no period after DOI

%\item Fraser D, Villaseñor A, Tóth A, Balk M, Eronen JT, {\bfseries Barr WA}, Behrensmeyer AK, Davis M, Du A, Faith JT, Gotelli NJ, Graves, G, Jukar AM, Looy CV, McGill BJ, Miller JH, Pineda-Munoz S, Potts R,  Shupinski AB, Soul LC, and Lyons SK. Profound Holocene biotic homogenization of North American mammalian faunas. In Review at \emph{Proceedings of the National Academy of Sciences}.

%\item Negash EW, Alemseged Z, {\bfseries Barr WA}, Behrensmeyer AK, Bobe R, Carvalho S, Wood BA, Wynn JG. Using modern African ecosystems as analogues to reconstruct woody cover and hominin environments. \emph{Journal of Human Evolution}. 

%\subsubsection*{In Revision}





\item Robinson JR, Rowan J,  {\bfseries Barr WA}, Sponheimer M. 2021. Intrataxonomic trends in herbivore enamel $\delta$13C are decoupled from ecosystem woody cover.  \emph{Nature Ecology and Evolution}. \href{https://dx.doi.org/10.1038/s41559-021-01455-7}{doi:10.1038/s41559-021-01455-7}

\item Geraads D, Reed D, {\bfseries Barr WA}, Bobe R, Stamos P, Alemseged Z. 2021 Plio-Pleistocene mammals from Mille-Logya, Ethiopia, and the post-Hadar faunal change. \emph{Journal of Quaternary Science}. 36:1073-1089.

\item Dumouchel L, Bobe R, Wynn J, {\bfseries Barr WA}. 2021. The environments of \emph{Australopithecus anamensis} at Allia Bay, Kenya: A multiproxy analysis of early Pliocene Bovidae. \emph{Journal of Human Evolution}. 151:102928. \href{https://doi.org/10.1016/j.jhevol.2020.102928}{doi:10.1016/ j.jhevol.2020.102928}




\item {\bfseries Barr WA}, Biernat M. 2020. Mammal functional diversity and habitat heterogeneity: Implications for hominin habitat reconstruction. \emph{Journal of Human Evolution}. 146:102853. \href{https://dx.doi.org/10.1016/j.jhevol.2020.102853}{doi:10.1016/j.jhevol.2020.102853}


\item Alemseged Z, Wynn JG, Geraads D, Reed DN, {\bfseries Barr WA}, Bobe R, McPherron S, Deino A, Alene M, Sier M, Roman D,  Mohan J. 2020. Fossils from Mille-Logya, Afar, Ethiopia, shed light on the link between late Pliocene environmental changes and the origin of \emph{Homo}. \emph{Nature Communications}. 11:2480. \href{https://doi.org/10.1038/s41467-020-16060-8 }{doi:10.1038/s41467-020-16060-8}




\item Patterson DB, Braun DR, Allen K, {\bfseries Barr WA}, Behrensmeyer AK, Biernat M, Lehmann SB, Maddox T, Manthi FK, Merritt SR, Morris SE, O'Brien K, Reeves JS, Wood BA, Bobe R. 2019. Comparative isotopic evidence from East Turkana is consistent with a dietary shift between early \emph{Homo} and \emph{Homo erectus}. \emph{Nature Ecology and Evolution}. 3:1048-1056. \href{https://dx.doi.org/10.1038/s41559-019-0916-0}{doi:10.1038/s41559-019-0916-0}



\item  Blondel C, Rowan J, Merceron G, Bibi F,  Negash E, {\bfseries Barr WA}, Boisserie JR. 2018. Feeding ecology of Tragelaphini (Bovidae) from the Shungura Formation, Omo Valley, Ethiopia: contribution of dental wear analyses.  \emph{Palaeogeography, Palaeoclimatology, Palaeoecology}. 496:103-120. \href{https://doi.org/10.1016/j.palaeo.2018.01.027}{doi:10.1016/j.palaeo.2018.01.027}



\item  {\bfseries Barr WA}. 2017. Bovid locomotor functional trait distributions reflect land cover and annual precipitation in sub-Saharan Africa. \emph{Evolutionary Ecology Research}.  \href{http://www.evolutionary-ecology.com/issues/v18/n03/ddar3051.pdf}{18:253-269}.


\item  {\bfseries Barr WA}. 2015. Paleoenvironments of the Shungura Formation (Plio-Pleistocene: Ethiopia) based on ecomorphology of the bovid astragalus. \emph{Journal of Human Evolution}. 88:97-107. \href{http://dx.doi.org/10.1016/j.jhevol.2015.05.002}{doi:10.1016/j.jhevol.2015.05.002}






\end{itemize}








\section*{Fieldwork }
\begin{longtable}{p{.16\textwidth}p{0.80\textwidth}}

2018 - Present & Tumbili Paleoanthropology Project, Laikipia County, Kenya (Late Miocene). I am the director of this new collaborative project in a unique geographical context outside the Great Rift Valley. To date we have discovered a moderately rich fossil fauna, and excavations are planned to expand the faunal sample from this poorly represented time period.\\[4pt]

2014 - Present & Mille-Logya Research Project, Afar Region, Ethiopia (Plio-Pleistocene). I conduct field research to recover new fossils and to understand the environmental and ecological context of human evolution in this region. \\[4pt]

2016 & Koobi-Fora Field School, East Turkana, Kenya. I collected fossil data relating to sub-regional faunal variability in the Koobi Fora Formation from 2.0 - 1.4 Ma. I supervised four undergraduate student research projects that were organized around this topic. \\[4pt]


2013 - 2014 & Great Divide Basin Project, Wyoming. Collected primate and mammalian fossils from Eocene sediments, and prospected for new localities. \\[4pt]

2010, 2012 & Dikika Research Project, Afar Region, Ethiopia. Surface collection of Plio- Pleistocene hominin and mammalian fossils. Managed GIS data collection with hand-held computers and high-precision GPS base station.\\[4pt]

2007, 2008, 2010 & Dalquest Research Site, Big Bend Region, Texas. Surface collected primate and mammalian fossils in the Devil's Graveyard Formation. (Eocene: Late Uintan).\\[4pt]

2009 & Contrebandiers Cave, Temara, Morocco.  Excavated site preserving Middle Stone Age archaeology (Aterian) and hominin remains. Performed systematic analysis of rodent fauna.\\
\end{longtable}


\end{document}
