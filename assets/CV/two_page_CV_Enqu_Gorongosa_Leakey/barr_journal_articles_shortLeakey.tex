%% For publications, include Citation, with period afterward. Then include DOI if available, with no period after DOI

%\item Fraser D, Villaseñor A, Tóth A, Balk M, Eronen JT, {\bfseries Barr WA}, Behrensmeyer AK, Davis M, Du A, Faith JT, Gotelli NJ, Graves, G, Jukar AM, Looy CV, McGill BJ, Miller JH, Pineda-Munoz S, Potts R,  Shupinski AB, Soul LC, and Lyons SK. Profound Holocene biotic homogenization of North American mammalian faunas. In Review at \emph{Proceedings of the National Academy of Sciences}.

%\item Negash EW, Alemseged Z, {\bfseries Barr WA}, Behrensmeyer AK, Bobe R, Carvalho S, Wood BA, Wynn JG. Using modern African ecosystems as analogues to reconstruct woody cover and hominin environments. \emph{Journal of Human Evolution}. 

%\subsubsection*{In Revision}





\item Robinson JR, Rowan J,  {\bfseries Barr WA}, Sponheimer M. 2021. Intrataxonomic trends in herbivore enamel $\delta$13C are decoupled from ecosystem woody cover.  \emph{Nature Ecology and Evolution}. \href{https://dx.doi.org/10.1038/s41559-021-01455-7}{doi:10.1038/s41559-021-01455-7}

\item Geraads D, Reed D, {\bfseries Barr WA}, Bobe R, Stamos P, Alemseged Z. 2021 Plio-Pleistocene mammals from Mille-Logya, Ethiopia, and the post-Hadar faunal change. \emph{Journal of Quaternary Science}. 36:1073-1089.

\item Dumouchel L, Bobe R, Wynn J, {\bfseries Barr WA}. 2021. The environments of \emph{Australopithecus anamensis} at Allia Bay, Kenya: A multiproxy analysis of early Pliocene Bovidae. \emph{Journal of Human Evolution}. 151:102928. \href{https://doi.org/10.1016/j.jhevol.2020.102928}{doi:10.1016/ j.jhevol.2020.102928}




\item {\bfseries Barr WA}, Biernat M. 2020. Mammal functional diversity and habitat heterogeneity: Implications for hominin habitat reconstruction. \emph{Journal of Human Evolution}. 146:102853. \href{https://dx.doi.org/10.1016/j.jhevol.2020.102853}{doi:10.1016/j.jhevol.2020.102853}


\item Alemseged Z, Wynn JG, Geraads D, Reed DN, {\bfseries Barr WA}, Bobe R, McPherron S, Deino A, Alene M, Sier M, Roman D,  Mohan J. 2020. Fossils from Mille-Logya, Afar, Ethiopia, shed light on the link between late Pliocene environmental changes and the origin of \emph{Homo}. \emph{Nature Communications}. 11:2480. \href{https://doi.org/10.1038/s41467-020-16060-8 }{doi:10.1038/s41467-020-16060-8}




\item Patterson DB, Braun DR, Allen K, {\bfseries Barr WA}, Behrensmeyer AK, Biernat M, Lehmann SB, Maddox T, Manthi FK, Merritt SR, Morris SE, O'Brien K, Reeves JS, Wood BA, Bobe R. 2019. Comparative isotopic evidence from East Turkana is consistent with a dietary shift between early \emph{Homo} and \emph{Homo erectus}. \emph{Nature Ecology and Evolution}. 3:1048-1056. \href{https://dx.doi.org/10.1038/s41559-019-0916-0}{doi:10.1038/s41559-019-0916-0}



\item  Blondel C, Rowan J, Merceron G, Bibi F,  Negash E, {\bfseries Barr WA}, Boisserie JR. 2018. Feeding ecology of Tragelaphini (Bovidae) from the Shungura Formation, Omo Valley, Ethiopia: contribution of dental wear analyses.  \emph{Palaeogeography, Palaeoclimatology, Palaeoecology}. 496:103-120. \href{https://doi.org/10.1016/j.palaeo.2018.01.027}{doi:10.1016/j.palaeo.2018.01.027}



\item  {\bfseries Barr WA}. 2017. Bovid locomotor functional trait distributions reflect land cover and annual precipitation in sub-Saharan Africa. \emph{Evolutionary Ecology Research}.  \href{http://www.evolutionary-ecology.com/issues/v18/n03/ddar3051.pdf}{18:253-269}.


\item  {\bfseries Barr WA}. 2015. Paleoenvironments of the Shungura Formation (Plio-Pleistocene: Ethiopia) based on ecomorphology of the bovid astragalus. \emph{Journal of Human Evolution}. 88:97-107. \href{http://dx.doi.org/10.1016/j.jhevol.2015.05.002}{doi:10.1016/j.jhevol.2015.05.002}





