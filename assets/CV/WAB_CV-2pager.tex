\documentclass{article}
\usepackage[top=0.9in, bottom=0.9in, left=0.85in, right=0.85in]{geometry}
\usepackage{mdwlist}
\usepackage{enumitem}
\usepackage{etaremune}
\usepackage{longtable}
\usepackage{verbatim}
\usepackage{fancyhdr}
\usepackage[colorlinks=True, urlcolor=cyan]{hyperref}
\usepackage{helvet}
\renewcommand{\familydefault}{\sfdefault}
\pagestyle{fancy}
\pagenumbering{arabic}
\lhead{\itshape W. Andrew Barr - CV}
\chead{}
\rhead{\thepage}
\lfoot{}
\cfoot{Updated \today}
\rfoot{}

\thispagestyle{empty}

%custom environment to control hanging indent in description
\newenvironment{mylist}
{\begin{description}[style=unboxed,leftmargin=1.3cm]}
{\end{description}}


\begin{document}
\begin{center}
\noindent{\bfseries{\Huge W. Andrew Barr - Curriculum Vitae}}
\end{center}

\vspace{15pt}

\noindent\begin{minipage}{.60\textwidth}
\begin{flushleft}
Department of Anthropology\\
Center for the Advanced Study of Human Paleobiology\\
The George Washington University\\
ORCID: \href{https://orcid.org/0000-0001-9763-6440}{0000-0001-9763-6440}\\
\end{flushleft}
\end{minipage}
\begin{minipage}{.395\textwidth}
\begin{flushright}
800 22nd St NW, Suite 6000\\
Washington, DC 20052 \\
wabarr@email.gwu.edu\\
+1 (202) 994-3213\\
\end{flushright}
\end{minipage}


\noindent\rule[-2mm]{\textwidth}{1pt}

\section*{Education}

\begin{tabular}{p{.1\textwidth}p{0.86\textwidth}}
2014 & University of Texas at Austin. Ph.D., Anthropology. \\[4pt] %\emph{The Paleoenvironments of Early Hominins in the Omo Shungura Formation (Plio-Pleistocene, Ethiopia): Synthesizing Multiple Lines of Evidence Using Phylogenetic Ecomorphology}.
2008 & University of Texas at Austin. M.A., Anthropology. \\[4pt]
2005 & Tulane University. B.S., Anthropology and French.\\
\end{tabular} 


\section*{Academic Appointments}
\begin{tabular}{p{.15\textwidth}p{0.8\textwidth}}
2019 - Present & Assistant Professor. Department of Anthropology. Center for the Advanced Study of Human Paleobiology. The George Washington University.\\[4pt]
2014 - Present & Research Associate. Department of Paleobiology.  National Museum of Natural History.\\[4pt]
2016 - 2019 & Visiting Assistant Professor. Department of Anthropology. Center for the Advanced Study of Human Paleobiology. The George Washington University.\\[4pt]
2014 - 2016 & Postdoctoral Scientist. Department of Anthropology. Center for the Advanced Study of Human Paleobiology. The George Washington University. Advisor: Bernard Wood.\\[4pt]
\end{tabular} 




\section*{Most Relevant Recent Grants}
\begin{tabular}{p{.1\textwidth}p{0.86\textwidth}}
2022 & Leakey Foundation - Modern African ecosystems as analogues for hominin paleo-landscapes. Role: Co-PI. \$19,876\\[4pt]
2021 & National Science Foundation - Examining the relationship between an increasingly carnivorous \emph{Homo erectus} and Pleistocene mammal extinctions. Role: PI. \$90,099\\[4pt]
2020 & National Science Foundation - Collaborative Research: Catching Fire: Pyrotechnology and Ecosystem Change in the Turkana Basin. Role: Co-PI. \$237,661.\\[4pt]
2019 &  National Science Foundation - Collaborative Research: REU Site: Past and Present Human-Environment Dynamics in the Turkana Basin, Kenya. Role: Senior Personnel. \$305,846\\[4pt]
2019 & National Science Foundation - HRRBAA: Paleontology and paleoanthropology of a potential Late Miocene site in the Laikipia highlands. Role: PI. \$26,581. 2020 fieldwork season postponed due to COVID-19. \\[4pt]


\end{tabular}

\section*{Most Relevant Recent Publications}

\begin{itemize} 




%\subsubsection*{In Press or Early View}


\item[] Negash E, {\bfseries Barr WA}. 2023. Relative abundance of grazing and browsing herbivores is not a direct reflection of vegetation structure: Implications for hominin paleoenvironmental reconstruction. \emph{Journal of Human Evolution}. 177:103328 \href{https://doi.org/10.1016/j.jhevol.2023.103328}{doi:10.1016/j.jhevol.2023.103328}


\item[]  {\bfseries Barr WA}, Pobiner BL, Rowan J, Du A, Faith JT.  2022. No sustained increase in zooarchaeological evidence for carnivory after the appearance of \emph{Homo erectus}. \emph{Proceedings of the National Academy of Sciences}. 119 (5) e2115540119. \href{https://doi.org/10.1073/pnas.2115540119}{doi:10.1073/pnas.2115540119}

\item [] Bobe R, Geraads D, Wynn JG, Reed D, {\bfseries Barr WA}, Alemseged Z. 2022. \emph{Fossil Vertebrates and Paleoenvironments of the Pliocene Hadar Formation at Dikika, Ethiopia}. In: Bobe R, Reynolds S (eds.) \emph{African Paleoecology and Human Evolution}. 229-241. Cambridge University Press, Cambridge. \href{https://doi.org/10.1017/9781139696470.019}{doi:10.1017/9781139696470.019}

\item[] Robinson JR, Rowan J,  {\bfseries Barr WA}, Sponheimer M. 2021. Intrataxonomic trends in herbivore enamel $\delta$13C are decoupled from ecosystem woody cover.  \emph{Nature Ecology and Evolution}. 5:995–1002. \href{https://dx.doi.org/10.1038/s41559-021-01455-7}{doi:10.1038/s41559-021-01455-7}

\item[] Geraads D, Reed D, {\bfseries Barr WA}, Bobe R, Stamos P, Alemseged Z. 2021/ Plio-Pleistocene mammals from Mille-Logya, Ethiopia, and the post-Hadar faunal change. \emph{Journal of Quaternary Science}. 36:1073-1089. \href{https://doi.org/10.1002/jqs.3345}{doi:10.1002/jqs.3345}

\item[] Dumouchel L, Bobe R, Wynn J, {\bfseries Barr WA}. 2021. The environments of \emph{Australopithecus anamensis} at Allia Bay, Kenya: A multiproxy analysis of early Pliocene Bovidae. \emph{Journal of Human Evolution}. 151:102928. \href{https://doi.org/10.1016/j.jhevol.2020.102928}{doi:10.1016/ j.jhevol.2020.102928}


\item[] {\bfseries Barr WA}, Biernat M. 2020. Mammal functional diversity and habitat heterogeneity: Implications for hominin habitat reconstruction. \emph{Journal of Human Evolution}. 146:102853. \href{https://dx.doi.org/10.1016/j.jhevol.2020.102853}{doi:10.1016/j.jhevol.2020.102853}

\item[] Faith JT, Rowan J, Du A, {\bfseries Barr WA}. 2020. The uncertain case for human-driven extinctions prior to \emph{Homo sapiens}. \emph{Quaternary Research}. 96:88-104. \href{https://dx.doi.org/10.1017/qua.2020.51}{doi:10.1017/qua.2020.51}

\item[] Alemseged Z, Wynn JG, Geraads D, Reed DN, {\bfseries Barr WA}, Bobe R, McPherron S, Deino A, Alene M, Sier M, Roman D,  Mohan J. 2020. Fossils from Mille-Logya, Afar, Ethiopia, shed light on the link between late Pliocene environmental changes and the origin of \emph{Homo}. \emph{Nature Communications}. 11:2480. \href{https://doi.org/10.1038/s41467-020-16060-8 }{doi:10.1038/s41467-020-16060-8}

\item[] Geraads D, Didier G, {\bfseries Barr WA}, Reed D, Laurin M. 2020. The fossil record of camelids demonstrates a late divergence between Bactrian camel and dromedary. \emph{Acta Palaeontologica Polonica}. 65(2):251-260. \href{https://doi.org/10.4202/app.00727.2020}{doi:10.4202/ app.00727.2020}


\item[]  Geraads D, {\bfseries Barr WA}, Reed DN, Laurin M, Alemseged Z.. 2019. New remains of \emph{Camelus grattardi} (Mammalia, Camelidae) from the Plio-Pleistocene of Ethiopia and the phylogeny of the genus. \emph{Journal of Mammalian Evolution}. \href{https://doi.org/10.1007/s10914-019-09489-2}{doi:10.1007/s10914-019-09489-2}


\item[] Patterson DB, Braun DR, Allen K, {\bfseries Barr WA}, Behrensmeyer AK, Biernat M, Lehmann SB, Maddox T, Manthi FK, Merritt SR, Morris SE, O'Brien K, Reeves JS, Wood BA, Bobe R. 2019. Comparative isotopic evidence from East Turkana is consistent with a dietary shift between early \emph{Homo} and \emph{Homo erectus}. \emph{Nature Ecology and Evolution}. 3:1048-1056. \href{https://dx.doi.org/10.1038/s41559-019-0916-0}{doi:10.1038/s41559-019-0916-0}

\item[]  Blondel C, Rowan J, Merceron G, Bibi F,  Negash E, {\bfseries Barr WA}, Boisserie JR. 2018. Feeding ecology of Tragelaphini (Bovidae) from the Shungura Formation, Omo Valley, Ethiopia: contribution of dental wear analyses.  \emph{Palaeogeography, Palaeoclimatology, Palaeoecology}. 496:103-120. \href{https://doi.org/10.1016/j.palaeo.2018.01.027}{doi:10.1016/j.palaeo.2018.01.027}

\item[]  {\bfseries Barr WA}. 2015. Paleoenvironments of the Shungura Formation (Plio-Pleistocene: Ethiopia) based on ecomorphology of the bovid astragalus. \emph{Journal of Human Evolution}. 88:97-107. \href{http://dx.doi.org/10.1016/j.jhevol.2015.05.002}{doi:10.1016/j.jhevol.2015.05.002}

\item[]  Reed D, {\bfseries Barr WA}, McPherron S, Bobe R, Geraads D, Wynn J, Alemseged Z. 2015. Digital Data Collection in Paleoanthropology. \emph{Evolutionary Anthropology}. 24:238-249. \href{http://dx.doi.org/10.1002/evan.21466}{doi:10.1002/evan.21466}

\item[] Thompson JC, McPherron S, Bobe R, Reed DN, {\bfseries Barr WA}, Wynn J, Marean CW, Geraads D, Alemseged Z. 2015. Taphonomy of fossils from the hominin-bearing deposits at Dikika, Ethiopia. \emph{Journal of Human Evolution}. 86:112-135. \href{http://dx.doi.org/10.1016/j.jhevol.2015.06.013}{doi:10.1016/j.jhevol.2015.06.013}

\end{itemize}




\end{document}
