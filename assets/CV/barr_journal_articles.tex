%% For publications, include Citation, with period afterward. Then include DOI if available, with no period after DOI



%\item Negash EW, Alemseged Z, {\bfseries Barr WA}, Behrensmeyer AK, Bobe R, Carvalho S, Wood BA, Wynn JG. Using modern African ecosystems as analogues to reconstruct woody cover and hominin environments. \emph{Journal of Human Evolution}. 

%\subsubsection*{In Revision}



%\subsubsection*{In Press or Early View}


\subsubsection*{2022}


\item {\bfseries Barr WA}, Pobiner BL, Rowan J, Du A, Faith JT.  No sustained increase in zooarchaeological evidence for carnivory after the appearance of \emph{Homo erectus}. \emph{Proceedings of the National Academy of Sciences}. 119 (5) e2115540119. \href{https://doi.org/10.1073/pnas.2115540119}{doi:10.1073/pnas.2115540119}

\item Fraser D, Villaseñor A, Tóth A, Balk M, Eronen JT, {\bfseries Barr WA}, Behrensmeyer AK, Davis M, Du A, Faith JT, Gotelli NJ, Graves, G, Jukar AM, Looy CV, McGill BJ, Miller JH, Pineda-Munoz S, Potts R,  Shupinski AB, Soul LC, and Lyons SK. Late Quaternary Biotic Homogenization of North American Mammalian Faunas. \emph{Nature Communications}. 13:3940. \href{https://doi.org/10.1038/s41467-022-31595-8}{doi:10.1038/s41467-022-31595-8}


\subsubsection*{2021}


\item Robinson JR, Rowan J,  {\bfseries Barr WA}, Sponheimer M. Intrataxonomic trends in herbivore enamel $\delta$13C are decoupled from ecosystem woody cover.  \emph{Nature Ecology and Evolution}. 5:995–1002. \href{https://dx.doi.org/10.1038/s41559-021-01455-7}{doi:10.1038/s41559-021-01455-7}

\item Geraads D, Reed D, {\bfseries Barr WA}, Bobe R, Stamos P, Alemseged Z. Plio-Pleistocene mammals from Mille-Logya, Ethiopia, and the post-Hadar faunal change. \emph{Journal of Quaternary Science}. 36:1073-1089. \href{https://doi.org/10.1002/jqs.3345}{doi:10.1002/jqs.3345}

\item Dumouchel L, Bobe R, Wynn J, {\bfseries Barr WA}. The environments of \emph{Australopithecus anamensis} at Allia Bay, Kenya: A multiproxy analysis of early Pliocene Bovidae. \emph{Journal of Human Evolution}. 151:102928. \href{https://doi.org/10.1016/j.jhevol.2020.102928}{doi:10.1016/ j.jhevol.2020.102928}

\item Fraser D, Soul LC, Tóth AB, Balk MA, Eronen JT, Pineda-Munoz S, Shupinski AB, Villaseñor A, {\bfseries Barr WA}, Behrensmeyer AK, Du A, Faith JT, Gotelli NJ, Graves GR, Jukar AM, Looy CV, Miller JH, Potts R, Lyons SK. Investigating biotic interactions in deep time. \emph{Trends in Ecology and Evolution}. 36:61-75, \href{https://doi.org/10.1016/j.tree.2020.09.001}{doi:10.1016/ j.tree.2020.09.001} 

\item Pineda-Munoz S, Jukar AM, Amatangelo K, Balk MA, {\bfseries Barr WA} Behrensmeyer AK, Blois J, Davis M, Du A, Eronen JT, Fraser D, Gotelli NJ, Looy C, Miller J, Shupinski AB, Soul LC, Tóth AB, Villaseñor A, Wing S, Lyons SK. Body mass-related changes in mammal community assembly patterns during the late Quaternary of North America. \emph{Ecography}. 44:56-66. \href{https://dx.doi.org/10.1111/ecog.05027}{doi:10.1111/ecog.05027}


\subsubsection*{2020}



\item {\bfseries Barr WA}, Biernat M. Mammal functional diversity and habitat heterogeneity: Implications for hominin habitat reconstruction. \emph{Journal of Human Evolution}. 146:102853. \href{https://dx.doi.org/10.1016/j.jhevol.2020.102853}{doi:10.1016/j.jhevol.2020.102853}

\item Faith JT, Rowan J, Du A, {\bfseries Barr WA}. The uncertain case for human-driven extinctions prior to \emph{Homo sapiens}. \emph{Quaternary Research}. 96:88-104. \href{https://dx.doi.org/10.1017/qua.2020.51}{doi:10.1017/qua.2020.51}

\item Alemseged Z, Wynn JG, Geraads D, Reed DN, {\bfseries Barr WA}, Bobe R, McPherron S, Deino A, Alene M, Sier M, Roman D,  Mohan J. Fossils from Mille-Logya, Afar, Ethiopia, shed light on the link between late Pliocene environmental changes and the origin of \emph{Homo}. \emph{Nature Communications}. 11:2480. \href{https://doi.org/10.1038/s41467-020-16060-8 }{doi:10.1038/s41467-020-16060-8}

\item Geraads D, Didier G, {\bfseries Barr WA}, Reed D, Laurin M. The fossil record of camelids demonstrates a late divergence between Bactrian camel and dromedary. \emph{Acta Palaeontologica Polonica}. 65(2):251-260. \href{https://doi.org/10.4202/app.00727.2020}{doi:10.4202/ app.00727.2020}

\item {\bfseries Barr WA}. The morphology of the bovid calcaneus: function, phylogenetic signal, and allometric scaling. \emph{Journal of Mammalian Evolution.}  27:111-121. \href{https://dx.doi.org/10.1007/s10914-018-9446-9}{doi:10.1007/s10914-018-9446-9}

\subsubsection*{2019}

\item Geraads D, {\bfseries Barr WA}, Reed DN, Laurin M, Alemseged Z.. New remains of \emph{Camelus grattardi} (Mammalia, Camelidae) from the Plio-Pleistocene of Ethiopia and the phylogeny of the genus. \emph{Journal of Mammalian Evolution}. \href{https://doi.org/10.1007/s10914-019-09489-2}{doi:10.1007/s10914-019-09489-2}

\item Tóth, AB, Lyons SK, {\bfseries Barr WA}, Behrensmeyer AK, Blois JL, Bobe R, Davis M, Du A, Eronen J, Faith JT, Fraser D, Gotelli NJ, Graves GR, Jukar AM, Miller JH, Pineda-Munoz S, Soul LC, Villaseñor A, Alroy J. Reorganization of surviving mammal communities after the end-Pleistocene megafaunal extinction. \emph{Science.} 365:1305-1308. \href{https://dx.doi.org/10.1126/science.aaw1605 }{doi:10.1126/science.aaw1605}

\item Patterson DB, Braun DR, Allen K, {\bfseries Barr WA}, Behrensmeyer AK, Biernat M, Lehmann SB, Maddox T, Manthi FK, Merritt SR, Morris SE, O'Brien K, Reeves JS, Wood BA, Bobe R. Comparative isotopic evidence from East Turkana is consistent with a dietary shift between early \emph{Homo} and \emph{Homo erectus}. \emph{Nature Ecology and Evolution}. 3:1048-1056. \href{https://dx.doi.org/10.1038/s41559-019-0916-0}{doi:10.1038/s41559-019-0916-0}

\subsubsection*{2018}

\item Fraser D, Haupt R,  {\bfseries Barr WA}. Phylogenetic signal in tooth wear dietary niche proxies: What it means for those in the field. \emph{Ecology and Evolution.} \href{https://dx.doi.org/10.1002/ece3.4540}{doi:10.1002/ece3.4540}


\item  Fraser D, Haupt R,  {\bfseries Barr WA}. Phylogenetic Signal In Tooth Wear Dietary Niche Proxies. \emph{Ecology and Evolution}. 8:5355-5368 \href{https://doi.org/10.1002/ece3.4052}{doi:10.1002/ece3.4052}

\item  Blondel C, Rowan J, Merceron G, Bibi F,  Negash E, {\bfseries Barr WA}, Boisserie JR. Feeding ecology of Tragelaphini (Bovidae) from the Shungura Formation, Omo Valley, Ethiopia: contribution of dental wear analyses.  \emph{Palaeogeography, Palaeoclimatology, Palaeoecology}. 496:103-120. \href{https://doi.org/10.1016/j.palaeo.2018.01.027}{doi:10.1016/j.palaeo.2018.01.027}

\subsubsection*{2017}

\item  {\bfseries Barr WA}. Signal or noise? A null model method for testing hypotheses about pulsed faunal turnover. \emph{Paleobiology}. 43:656-666. \href{https://doi.org/10.1017/pab.2017.21}{doi:10.1017/pab.2017.21}

\item  {\bfseries Barr WA}. Bovid locomotor functional trait distributions reflect land cover and annual precipitation in sub-Saharan Africa. \emph{Evolutionary Ecology Research}.  \href{http://www.evolutionary-ecology.com/issues/v18/n03/ddar3051.pdf}{18:253-269}.

\subsubsection*{2015}

\item  {\bfseries Barr WA}. Paleoenvironments of the Shungura Formation (Plio-Pleistocene: Ethiopia) based on ecomorphology of the bovid astragalus. \emph{Journal of Human Evolution}. 88:97-107. \href{http://dx.doi.org/10.1016/j.jhevol.2015.05.002}{doi:10.1016/j.jhevol.2015.05.002}

\item  Reed D, {\bfseries Barr WA}, McPherron S, Bobe R, Geraads D, Wynn J, Alemseged Z. Digital Data Collection in Paleoanthropology. \emph{Evolutionary Anthropology}. 24:238-249. \href{http://dx.doi.org/10.1002/evan.21466}{doi:10.1002/evan.21466}

\item  Thompson JC, McPherron S, Bobe R, Reed DN, {\bfseries Barr WA}, Wynn J, Marean CW, Geraads D, Alemseged Z. Taphonomy of fossils from the hominin-bearing deposits at Dikika, Ethiopia. \emph{Journal of Human Evolution}. 86:112-135. \href{http://dx.doi.org/10.1016/j.jhevol.2015.06.013}{doi:10.1016/j.jhevol.2015.06.013}

\subsubsection*{2014}

\item  {\bfseries Barr WA}. Functional Morphology of the Bovid Astragalus In Relation To Habitat: Controlling Phylogenetic Signal In Ecomorphology. \emph{Journal of Morphology}. 275:1201-1216. \href{http://dx.doi.org/10.1002/jmor.20279}{doi:10.1002/jmor.20279}

\item  {\bfseries Barr WA} and Scott RS. Phylogenetic comparative methods complement discriminant function analysis in ecomorphology. \emph{American Journal of Physical Anthropology}. 153:663-674. \href{http://dx.doi.org/10.1002/ajpa.22462}{doi:10.1002/ajpa.22462}

\item  Scott RS and {\bfseries Barr WA}. Ecomorphology and phylogenetic risk: implications for habitat reconstruction using fossil bovids. \emph{Journal of Human Evolution}. 73:47-57. \href{http://dx.doi.org/10.1016/j.jhevol.2014.02.023}{doi:10.1016/j.jhevol.2014.02.023}

\subsubsection*{2010}

\item  Reed DN, and {\bfseries Barr WA}. A preliminary account of the rodents from Pleistocene levels at Grotte des Contrebandiers (Smuggler's Cave), Morocco. \emph{Historical Biology}. 22:286-294. \href{http://dx.doi.org/10.1080/08912960903562192}{doi:10.1080/08912960903562192}




