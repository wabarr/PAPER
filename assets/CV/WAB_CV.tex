\documentclass{article}
\usepackage[top=0.7in, bottom=0.7in, left=0.7in, right=0.7in]{geometry}
\usepackage{mdwlist}
\usepackage{enumitem}
\usepackage{etaremune}
\usepackage{longtable}
\usepackage{verbatim}
\usepackage{fancyhdr}
\usepackage[colorlinks=True, urlcolor=cyan]{hyperref}
\usepackage{helvet}
\renewcommand{\familydefault}{\sfdefault}
\pagestyle{fancy}
\pagenumbering{arabic}
\lhead{\itshape W. Andrew Barr - CV}
\chead{}
\rhead{\thepage}
\lfoot{}
\cfoot{Updated \today}
\rfoot{}

\thispagestyle{empty}

%custom environment to control hanging indent in description
\newenvironment{mylist}
{\begin{description}[style=unboxed,leftmargin=1.3cm]}
{\end{description}}


\begin{document}
\begin{center}
\noindent{\bfseries{\Huge W. Andrew Barr - Curriculum Vitae}}
\end{center}

\vspace{15pt}

\noindent\begin{minipage}{.60\textwidth}
\begin{flushleft}
Department of Anthropology\\
Center for the Advanced Study of Human Paleobiology\\
The George Washington University\\
ORCID: \href{https://orcid.org/0000-0001-9763-6440}{0000-0001-9763-6440}\\
\end{flushleft}
\end{minipage}
\begin{minipage}{.395\textwidth}
\begin{flushright}
800 22nd St NW, Suite 6000\\
Washington, DC 20052 \\
wabarr@email.gwu.edu\\
+1 (202) 994-3213\\
\end{flushright}
\end{minipage}


\noindent\rule[-2mm]{\textwidth}{1pt}

\section*{Education}

\begin{tabular}{p{.1\textwidth}p{0.86\textwidth}}
2014 & University of Texas at Austin. Ph.D., Anthropology. \\[4pt] %\emph{The Paleoenvironments of Early Hominins in the Omo Shungura Formation (Plio-Pleistocene, Ethiopia): Synthesizing Multiple Lines of Evidence Using Phylogenetic Ecomorphology}.
2008 & University of Texas at Austin. M.A., Anthropology. \\[4pt]
2005 & Tulane University. B.S., Anthropology and French.\\
\end{tabular} 


\section*{Academic Appointments}
\begin{tabular}{p{.15\textwidth}p{0.8\textwidth}}
2019 - Present & Assistant Professor. Department of Anthropology. Center for the Advanced Study of Human Paleobiology. The George Washington University.\\[4pt]
2014 - Present & Research Associate. Department of Paleobiology.  National Museum of Natural History.\\[4pt]
2016 - 2019 & Visiting Assistant Professor. Department of Anthropology. Center for the Advanced Study of Human Paleobiology. The George Washington University.\\[4pt]
2014 - 2016 & Postdoctoral Scientist. Department of Anthropology. Center for the Advanced Study of Human Paleobiology. The George Washington University. \\[4pt]
\end{tabular} 


\section*{Research Grants and Fellowships}
\begin{tabular}{p{.1\textwidth}p{0.86\textwidth}}
2022 & Leakey Foundation - Modern African ecosystems as analogues for hominin paleo-landscapes. Role: Co-PI. \$19,876\\[4pt]
2021 & National Science Foundation - Examining the relationship between an increasingly carnivorous \emph{Homo erectus} and Pleistocene mammal extinctions. Role: PI. \$90,099\\[4pt]
2020 & National Science Foundation - Collaborative Research: Catching Fire: Pyrotechnology and Ecosystem Change in the Turkana Basin. Role: Co-PI. \$237,661.\\[4pt]
2019 &  National Science Foundation - Collaborative Research: REU Site: Past and Present Human-Environment Dynamics in the Turkana Basin, Kenya. Role: Senior Personnel. \$305,846\\[4pt]
2019 & National Science Foundation - HRRBAA: Paleontology and paleoanthropology of a potential Late Miocene site in the Laikipia highlands. Role: PI. \$26,581. 2020 and 2021 fieldwork seasons postponed due to COVID-19. \\[4pt]
2013 &  University of Texas at Austin - Named Continuing Fellowship. \$29,000.\\[4pt]
2012  & Wenner-Gren Foundation - Dissertation Fieldwork Grant. \$13,317.\\[4pt]
2007 &  National Science Foundation - Graduate Research Fellowship. \$90,000.\\[4pt]
%2007 &  University of Texas at Austin - Liberal Arts Graduate Research Fellowship. \$1,800.\\[4pt]
%2007, 2008 &  David Bruton, Jr. Graduate Fellowship - University of Texas at Austin. \$1,000.\\[4pt]

\end{tabular}

\section*{Peer-Reviewed Journal Articles}

%*For all first-authored papers, I designed the research question, collected data, analyzed data, and wrote the paper. On papers for which I am a middle author, I played an important role in either research design and/or data analysis, and the writing of the paper.  For last-authored papers I played a supervisory role in designing the research question and/or analyzing the data, and contributed to writing the paper. 

\begin{etaremune} 





\item[] In Press - Powell V, {\bfseries Barr WA}, Hammond AS, Wood BA. Behavioral and phylogenetic correlates of limb length proportions in extant apes and monkeys: implications for interpreting fossil associated skeletons. \emph{Journal of Human Evolution}.

\item[] In Review - {\bfseries Barr WA}, Wood BA. Thinking outside the rift: Investigating how spatial bias influences our understanding of early hominin evolution. \emph{Nature Ecology and Evolution}.


\item 2023. Negash E, {\bfseries Barr WA}. Relative abundance of grazing and browsing herbivores is not a direct reflection of vegetation structure: Implications for hominin paleoenvironmental reconstruction. \emph{Journal of Human Evolution}. 177:103328 \href{https://doi.org/10.1016/j.jhevol.2023.103328}{doi:10.1016/j.jhevol.2023.103328}

\item 2022. Bobe R, Geraads D, Wynn JG, Reed D, {\bfseries Barr WA}, Alemseged Z. \emph{Fossil Vertebrates and Paleoenvironments of the Pliocene Hadar Formation at Dikika, Ethiopia}. In: Bobe R, Reynolds S (eds.) \emph{African Paleoecology and Human Evolution}. 229-241. Cambridge University Press, Cambridge. \href{https://doi.org/10.1017/9781139696470.019}{doi:10.1017/9781139696470.019}

\item 2022. {\bfseries Barr WA}, Pobiner BL, Rowan J, Du A, Faith JT.  No sustained increase in zooarchaeological evidence for carnivory after the appearance of \emph{Homo erectus}. \emph{Proceedings of the National Academy of Sciences}. 119 (5) e2115540119. \href{https://doi.org/10.1073/pnas.2115540119}{doi:10.1073/pnas.2115540119}

\item 2022. Fraser D, Villaseñor A, Tóth A, Balk M, Eronen JT, {\bfseries Barr WA}, Behrensmeyer AK, Davis M, Du A, Faith JT, Gotelli NJ, Graves, G, Jukar AM, Looy CV, McGill BJ, Miller JH, Pineda-Munoz S, Potts R,  Shupinski AB, Soul LC, and Lyons SK. Late Quaternary Biotic Homogenization of North American Mammalian Faunas. \emph{Nature Communications}. 13:3940. \href{https://doi.org/10.1038/s41467-022-31595-8}{doi:10.1038/s41467-022-31595-8}



\item 2021. Robinson JR, Rowan J,  {\bfseries Barr WA}, Sponheimer M. Intrataxonomic trends in herbivore enamel $\delta$13C are decoupled from ecosystem woody cover.  \emph{Nature Ecology and Evolution}. 5:995–1002. \href{https://dx.doi.org/10.1038/s41559-021-01455-7}{doi:10.1038/s41559-021-01455-7}

\item 2021. Geraads D, Reed D, {\bfseries Barr WA}, Bobe R, Stamos P, Alemseged Z. Plio-Pleistocene mammals from Mille-Logya, Ethiopia, and the post-Hadar faunal change. \emph{Journal of Quaternary Science}. 36:1073-1089. \href{https://doi.org/10.1002/jqs.3345}{doi:10.1002/jqs.3345}

\item 2021. Dumouchel L, Bobe R, Wynn J, {\bfseries Barr WA}. The environments of \emph{Australopithecus anamensis} at Allia Bay, Kenya: A multiproxy analysis of early Pliocene Bovidae. \emph{Journal of Human Evolution}. 151:102928. \href{https://doi.org/10.1016/j.jhevol.2020.102928}{doi:10.1016/ j.jhevol.2020.102928}

\item 2021. Fraser D, Soul LC, Tóth AB, Balk MA, Eronen JT, Pineda-Munoz S, Shupinski AB, Villaseñor A, {\bfseries Barr WA}, Behrensmeyer AK, Du A, Faith JT, Gotelli NJ, Graves GR, Jukar AM, Looy CV, Miller JH, Potts R, Lyons SK. Investigating biotic interactions in deep time. \emph{Trends in Ecology and Evolution}. 36:61-75, \href{https://doi.org/10.1016/j.tree.2020.09.001}{doi:10.1016/ j.tree.2020.09.001} 

\item 2021. Pineda-Munoz S, Jukar AM, Amatangelo K, Balk MA, {\bfseries Barr WA} Behrensmeyer AK, Blois J, Davis M, Du A, Eronen JT, Fraser D, Gotelli NJ, Looy C, Miller J, Shupinski AB, Soul LC, Tóth AB, Villaseñor A, Wing S, Lyons SK. Body mass-related changes in mammal community assembly patterns during the late Quaternary of North America. \emph{Ecography}. 44:56-66. \href{https://dx.doi.org/10.1111/ecog.05027}{doi:10.1111/ecog.05027}



\item 2020. {\bfseries Barr WA}, Biernat M. Mammal functional diversity and habitat heterogeneity: Implications for hominin habitat reconstruction. \emph{Journal of Human Evolution}. 146:102853. \href{https://dx.doi.org/10.1016/j.jhevol.2020.102853}{doi:10.1016/j.jhevol.2020.102853}

\item 2020. Faith JT, Rowan J, Du A, {\bfseries Barr WA}. The uncertain case for human-driven extinctions prior to \emph{Homo sapiens}. \emph{Quaternary Research}. 96:88-104. \href{https://dx.doi.org/10.1017/qua.2020.51}{doi:10.1017/qua.2020.51}

\item 2020. Alemseged Z, Wynn JG, Geraads D, Reed DN, {\bfseries Barr WA}, Bobe R, McPherron S, Deino A, Alene M, Sier M, Roman D,  Mohan J. Fossils from Mille-Logya, Afar, Ethiopia, shed light on the link between late Pliocene environmental changes and the origin of \emph{Homo}. \emph{Nature Communications}. 11:2480. \href{https://doi.org/10.1038/s41467-020-16060-8 }{doi:10.1038/s41467-020-16060-8}

\item 2020. Geraads D, Didier G, {\bfseries Barr WA}, Reed D, Laurin M. The fossil record of camelids demonstrates a late divergence between Bactrian camel and dromedary. \emph{Acta Palaeontologica Polonica}. 65(2):251-260. \href{https://doi.org/10.4202/app.00727.2020}{doi:10.4202/ app.00727.2020}

\item 2020. {\bfseries Barr WA}. The morphology of the bovid calcaneus: function, phylogenetic signal, and allometric scaling. \emph{Journal of Mammalian Evolution.}  27:111-121. \href{https://dx.doi.org/10.1007/s10914-018-9446-9}{doi:10.1007/s10914-018-9446-9}


\item 2019. Geraads D, {\bfseries Barr WA}, Reed DN, Laurin M, Alemseged Z.. New remains of \emph{Camelus grattardi} (Mammalia, Camelidae) from the Plio-Pleistocene of Ethiopia and the phylogeny of the genus. \emph{Journal of Mammalian Evolution}. \href{https://doi.org/10.1007/s10914-019-09489-2}{doi:10.1007/s10914-019-09489-2}

\item 2019. Tóth, AB, Lyons SK, {\bfseries Barr WA}, Behrensmeyer AK, Blois JL, Bobe R, Davis M, Du A, Eronen J, Faith JT, Fraser D, Gotelli NJ, Graves GR, Jukar AM, Miller JH, Pineda-Munoz S, Soul LC, Villaseñor A, Alroy J. Reorganization of surviving mammal communities after the end-Pleistocene megafaunal extinction. \emph{Science.} 365:1305-1308. \href{https://dx.doi.org/10.1126/science.aaw1605 }{doi:10.1126/science.aaw1605}

\item 2019. Patterson DB, Braun DR, Allen K, {\bfseries Barr WA}, Behrensmeyer AK, Biernat M, Lehmann SB, Maddox T, Manthi FK, Merritt SR, Morris SE, O'Brien K, Reeves JS, Wood BA, Bobe R. Comparative isotopic evidence from East Turkana is consistent with a dietary shift between early \emph{Homo} and \emph{Homo erectus}. \emph{Nature Ecology and Evolution}. 3:1048-1056. \href{https://dx.doi.org/10.1038/s41559-019-0916-0}{doi:10.1038/s41559-019-0916-0}

\item 2018. {\bfseries Barr WA}. \emph{Ecomorphology}. In: Croft DA, Simpson SW, and Su DF (eds.), \emph{Methods in Paleoecology: Reconstructing Cenozoic Terrestrial Environments and Ecological Communities}. 339-349. Springer (Vertebrate Paleobiology and Paleoanthropology Series), Cham, Switzerland. \href{https://doi.org/10.1007/978-3-319-94265-0}{doi:10.1007/978-3-319-94265-0}

\item 2018. Reed, DN, {\bfseries Barr WA}, Kappelman J. \emph{PaleoCore: an open-source platform for geospatial data integration in paleoanthropology}. In: Anemone R, Conroy G (eds.), \emph{New Geospatial Approaches in Anthropology}. University of New Mexico Press. Albuquerque, NM.

\item 2018. Fraser D, Haupt R,  {\bfseries Barr WA}. Phylogenetic signal in tooth wear dietary niche proxies: What it means for those in the field. \emph{Ecology and Evolution.} \href{https://dx.doi.org/10.1002/ece3.4540}{doi:10.1002/ece3.4540}


\item  2018. Fraser D, Haupt R,  {\bfseries Barr WA}. Phylogenetic Signal In Tooth Wear Dietary Niche Proxies. \emph{Ecology and Evolution}. 8:5355-5368 \href{https://doi.org/10.1002/ece3.4052}{doi:10.1002/ece3.4052}

\item  2018. Blondel C, Rowan J, Merceron G, Bibi F,  Negash E, {\bfseries Barr WA}, Boisserie JR. Feeding ecology of Tragelaphini (Bovidae) from the Shungura Formation, Omo Valley, Ethiopia: contribution of dental wear analyses.  \emph{Palaeogeography, Palaeoclimatology, Palaeoecology}. 496:103-120. \href{https://doi.org/10.1016/j.palaeo.2018.01.027}{doi:10.1016/j.palaeo.2018.01.027}


\item  2017. {\bfseries Barr WA}. Signal or noise? A null model method for testing hypotheses about pulsed faunal turnover. \emph{Paleobiology}. 43:656-666. \href{https://doi.org/10.1017/pab.2017.21}{doi:10.1017/pab.2017.21}

\item  2017. {\bfseries Barr WA}. Bovid locomotor functional trait distributions reflect land cover and annual precipitation in sub-Saharan Africa. \emph{Evolutionary Ecology Research}.  \href{http://www.evolutionary-ecology.com/issues/v18/n03/ddar3051.pdf}{18:253-269}.


\item  2015. {\bfseries Barr WA}. Paleoenvironments of the Shungura Formation (Plio-Pleistocene: Ethiopia) based on ecomorphology of the bovid astragalus. \emph{Journal of Human Evolution}. 88:97-107. \href{http://dx.doi.org/10.1016/j.jhevol.2015.05.002}{doi:10.1016/j.jhevol.2015.05.002}

\item  2015. Reed D, {\bfseries Barr WA}, McPherron S, Bobe R, Geraads D, Wynn J, Alemseged Z. Digital Data Collection in Paleoanthropology. \emph{Evolutionary Anthropology}. 24:238-249. \href{http://dx.doi.org/10.1002/evan.21466}{doi:10.1002/evan.21466}

\item  2015. Thompson JC, McPherron S, Bobe R, Reed DN, {\bfseries Barr WA}, Wynn J, Marean CW, Geraads D, Alemseged Z. Taphonomy of fossils from the hominin-bearing deposits at Dikika, Ethiopia. \emph{Journal of Human Evolution}. 86:112-135. \href{http://dx.doi.org/10.1016/j.jhevol.2015.06.013}{doi:10.1016/j.jhevol.2015.06.013}



\item  2014. {\bfseries Barr WA}. Functional Morphology of the Bovid Astragalus In Relation To Habitat: Controlling Phylogenetic Signal In Ecomorphology. \emph{Journal of Morphology}. 275:1201-1216. \href{http://dx.doi.org/10.1002/jmor.20279}{doi:10.1002/jmor.20279}

\item  2014. {\bfseries Barr WA} and Scott RS. Phylogenetic comparative methods complement discriminant function analysis in ecomorphology. \emph{American Journal of Physical Anthropology}. 153:663-674. \href{http://dx.doi.org/10.1002/ajpa.22462}{doi:10.1002/ajpa.22462}

\item  2014. Scott RS and {\bfseries Barr WA}. Ecomorphology and phylogenetic risk: implications for habitat reconstruction using fossil bovids. \emph{Journal of Human Evolution}. 73:47-57. \href{http://dx.doi.org/10.1016/j.jhevol.2014.02.023}{doi:10.1016/j.jhevol.2014.02.023}


\item  2010. Reed DN, and {\bfseries Barr WA}. A preliminary account of the rodents from Pleistocene levels at Grotte des Contrebandiers (Smuggler's Cave), Morocco. \emph{Historical Biology}. 22:286-294. \href{http://dx.doi.org/10.1080/08912960903562192}{doi:10.1080/08912960903562192}


\end{etaremune}

%%% For publications, include Citation, with period afterward. Then include DOI if available, with no period after DOI

%\item Fraser D, Villaseñor A, Tóth A, Balk M, Eronen JT, {\bfseries Barr WA}, Behrensmeyer AK, Davis M, Du A, Faith JT, Gotelli NJ, Graves, G, Jukar AM, Looy CV, McGill BJ, Miller JH, Pineda-Munoz S, Potts R,  Shupinski AB, Soul LC, and Lyons SK. Profound Holocene biotic homogenization of North American mammalian faunas. In Review at \emph{Proceedings of the National Academy of Sciences}.

%\item Negash EW, Alemseged Z, {\bfseries Barr WA}, Behrensmeyer AK, Bobe R, Carvalho S, Wood BA, Wynn JG. Using modern African ecosystems as analogues to reconstruct woody cover and hominin environments. \emph{Journal of Human Evolution}. 

%\subsubsection*{In Revision}



\subsubsection*{In Press or Early View}

\item {\bfseries Barr WA}, Pobiner BL, Rowan J, Du A, Faith JT.  No sustained increase in zooarchaeological evidence for carnivory after the appearance of \emph{Homo erectus}. \emph{Proceedings of the National Academy of Sciences}.

\subsubsection*{2021}

\item Robinson JR, Rowan J,  {\bfseries Barr WA}, Sponheimer M. Intrataxonomic trends in herbivore enamel $\delta$13C are decoupled from ecosystem woody cover.  \emph{Nature Ecology and Evolution}. \href{https://dx.doi.org/10.1038/s41559-021-01455-7}{doi:10.1038/s41559-021-01455-7}

\item Geraads D, Reed D, {\bfseries Barr WA}, Bobe R, Stamos P, Alemseged Z. Plio-Pleistocene mammals from Mille-Logya, Ethiopia, and the post-Hadar faunal change. \emph{Journal of Quaternary Science}. 36:1073-1089.

\item Dumouchel L, Bobe R, Wynn J, {\bfseries Barr WA}. The environments of \emph{Australopithecus anamensis} at Allia Bay, Kenya: A multiproxy analysis of early Pliocene Bovidae. \emph{Journal of Human Evolution}. 151:102928. \href{https://doi.org/10.1016/j.jhevol.2020.102928}{doi:10.1016/ j.jhevol.2020.102928}

\item Fraser D, Soul LC, Tóth AB, Balk MA, Eronen JT, Pineda-Munoz S, Shupinski AB, Villaseñor A, {\bfseries Barr WA}, Behrensmeyer AK, Du A, Faith JT, Gotelli NJ, Graves GR, Jukar AM, Looy CV, Miller JH, Potts R, Lyons SK. Investigating biotic interactions in deep time. \emph{Trends in Ecology and Evolution}. 36:61-75, \href{https://doi.org/10.1016/j.tree.2020.09.001}{doi:10.1016/ j.tree.2020.09.001} 

\item Pineda-Munoz S, Jukar AM, Amatangelo K, Balk MA, {\bfseries Barr WA} Behrensmeyer AK, Blois J, Davis M, Du A, Eronen JT, Fraser D, Gotelli NJ, Looy C, Miller J, Shupinski AB, Soul LC, Tóth AB, Villaseñor A, Wing S, Lyons SK. Body mass-related changes in mammal community assembly patterns during the late Quaternary of North America. \emph{Ecography}. 44:56-66. \href{https://dx.doi.org/10.1111/ecog.05027}{doi:10.1111/ecog.05027}


\subsubsection*{2020}



\item {\bfseries Barr WA}, Biernat M. Mammal functional diversity and habitat heterogeneity: Implications for hominin habitat reconstruction. \emph{Journal of Human Evolution}. 146:102853. \href{https://dx.doi.org/10.1016/j.jhevol.2020.102853}{doi:10.1016/j.jhevol.2020.102853}

\item Faith JT, Rowan J, Du A, {\bfseries Barr WA}. The uncertain case for human-driven extinctions prior to \emph{Homo sapiens}. \emph{Quaternary Research}. 96:88-104. \href{https://dx.doi.org/10.1017/qua.2020.51}{doi:10.1017/qua.2020.51}

\item Alemseged Z, Wynn JG, Geraads D, Reed DN, {\bfseries Barr WA}, Bobe R, McPherron S, Deino A, Alene M, Sier M, Roman D,  Mohan J. Fossils from Mille-Logya, Afar, Ethiopia, shed light on the link between late Pliocene environmental changes and the origin of \emph{Homo}. \emph{Nature Communications}. 11:2480. \href{https://doi.org/10.1038/s41467-020-16060-8 }{doi:10.1038/s41467-020-16060-8}

\item Geraads D, Didier G, {\bfseries Barr WA}, Reed D, Laurin M. The fossil record of camelids demonstrates a late divergence between Bactrian camel and dromedary. \emph{Acta Palaeontologica Polonica}. 65(2):251-260. \href{https://doi.org/10.4202/app.00727.2020}{doi:10.4202/ app.00727.2020}

\item {\bfseries Barr WA}. The morphology of the bovid calcaneus: function, phylogenetic signal, and allometric scaling. \emph{Journal of Mammalian Evolution.}  27:111-121. \href{https://dx.doi.org/10.1007/s10914-018-9446-9}{doi:10.1007/s10914-018-9446-9}

\subsubsection*{2019}

\item Geraads D, {\bfseries Barr WA}, Reed DN, Laurin M, Alemseged Z.. New remains of \emph{Camelus grattardi} (Mammalia, Camelidae) from the Plio-Pleistocene of Ethiopia and the phylogeny of the genus. \emph{Journal of Mammalian Evolution}. \href{https://doi.org/10.1007/s10914-019-09489-2}{doi:10.1007/s10914-019-09489-2}

\item Tóth, AB, Lyons SK, {\bfseries Barr WA}, Behrensmeyer AK, Blois JL, Bobe R, Davis M, Du A, Eronen J, Faith JT, Fraser D, Gotelli NJ, Graves GR, Jukar AM, Miller JH, Pineda-Munoz S, Soul LC, Villaseñor A, Alroy J. Reorganization of surviving mammal communities after the end-Pleistocene megafaunal extinction. \emph{Science.} 365:1305-1308. \href{https://dx.doi.org/10.1126/science.aaw1605 }{doi:10.1126/science.aaw1605}

\item Patterson DB, Braun DR, Allen K, {\bfseries Barr WA}, Behrensmeyer AK, Biernat M, Lehmann SB, Maddox T, Manthi FK, Merritt SR, Morris SE, O'Brien K, Reeves JS, Wood BA, Bobe R. Comparative isotopic evidence from East Turkana is consistent with a dietary shift between early \emph{Homo} and \emph{Homo erectus}. \emph{Nature Ecology and Evolution}. 3:1048-1056. \href{https://dx.doi.org/10.1038/s41559-019-0916-0}{doi:10.1038/s41559-019-0916-0}

\subsubsection*{2018}

\item Fraser D, Haupt R,  {\bfseries Barr WA}. Phylogenetic signal in tooth wear dietary niche proxies: What it means for those in the field. \emph{Ecology and Evolution.} \href{https://dx.doi.org/10.1002/ece3.4540}{doi:10.1002/ece3.4540}


\item  Fraser D, Haupt R,  {\bfseries Barr WA}. Phylogenetic Signal In Tooth Wear Dietary Niche Proxies. \emph{Ecology and Evolution}. 8:5355-5368 \href{https://doi.org/10.1002/ece3.4052}{doi:10.1002/ece3.4052}

\item  Blondel C, Rowan J, Merceron G, Bibi F,  Negash E, {\bfseries Barr WA}, Boisserie JR. Feeding ecology of Tragelaphini (Bovidae) from the Shungura Formation, Omo Valley, Ethiopia: contribution of dental wear analyses.  \emph{Palaeogeography, Palaeoclimatology, Palaeoecology}. 496:103-120. \href{https://doi.org/10.1016/j.palaeo.2018.01.027}{doi:10.1016/j.palaeo.2018.01.027}

\subsubsection*{2017}

\item  {\bfseries Barr WA}. Signal or noise? A null model method for testing hypotheses about pulsed faunal turnover. \emph{Paleobiology}. 43:656-666. \href{https://doi.org/10.1017/pab.2017.21}{doi:10.1017/pab.2017.21}

\item  {\bfseries Barr WA}. Bovid locomotor functional trait distributions reflect land cover and annual precipitation in sub-Saharan Africa. \emph{Evolutionary Ecology Research}.  \href{http://www.evolutionary-ecology.com/issues/v18/n03/ddar3051.pdf}{18:253-269}.

\subsubsection*{2015}

\item  {\bfseries Barr WA}. Paleoenvironments of the Shungura Formation (Plio-Pleistocene: Ethiopia) based on ecomorphology of the bovid astragalus. \emph{Journal of Human Evolution}. 88:97-107. \href{http://dx.doi.org/10.1016/j.jhevol.2015.05.002}{doi:10.1016/j.jhevol.2015.05.002}

\item  Reed D, {\bfseries Barr WA}, McPherron S, Bobe R, Geraads D, Wynn J, Alemseged Z. Digital Data Collection in Paleoanthropology. \emph{Evolutionary Anthropology}. 24:238-249. \href{http://dx.doi.org/10.1002/evan.21466}{doi:10.1002/evan.21466}

\item  Thompson JC, McPherron S, Bobe R, Reed DN, {\bfseries Barr WA}, Wynn J, Marean CW, Geraads D, Alemseged Z. Taphonomy of fossils from the hominin-bearing deposits at Dikika, Ethiopia. \emph{Journal of Human Evolution}. 86:112-135. \href{http://dx.doi.org/10.1016/j.jhevol.2015.06.013}{doi:10.1016/j.jhevol.2015.06.013}

\subsubsection*{2014}

\item  {\bfseries Barr WA}. Functional Morphology of the Bovid Astragalus In Relation To Habitat: Controlling Phylogenetic Signal In Ecomorphology. \emph{Journal of Morphology}. 275:1201-1216. \href{http://dx.doi.org/10.1002/jmor.20279}{doi:10.1002/jmor.20279}

\item  {\bfseries Barr WA} and Scott RS. Phylogenetic comparative methods complement discriminant function analysis in ecomorphology. \emph{American Journal of Physical Anthropology}. 153:663-674. \href{http://dx.doi.org/10.1002/ajpa.22462}{doi:10.1002/ajpa.22462}

\item  Scott RS and {\bfseries Barr WA}. Ecomorphology and phylogenetic risk: implications for habitat reconstruction using fossil bovids. \emph{Journal of Human Evolution}. 73:47-57. \href{http://dx.doi.org/10.1016/j.jhevol.2014.02.023}{doi:10.1016/j.jhevol.2014.02.023}

\subsubsection*{2010}

\item  Reed DN, and {\bfseries Barr WA}. A preliminary account of the rodents from Pleistocene levels at Grotte des Contrebandiers (Smuggler's Cave), Morocco. \emph{Historical Biology}. 22:286-294. \href{http://dx.doi.org/10.1080/08912960903562192}{doi:10.1080/08912960903562192}







%\section*{Peer-Reviewed Book Chapters}
%\begin{etaremune}
%\subsubsection*{2022}

\item Bobe R, Geraads D, Wynn JG, Reed D, {\bfseries Barr WA}, Alemseged Z. \emph{Fossil Vertebrates and Paleoenvironments of the Pliocene Hadar Formation at Dikika, Ethiopia}. In: Bobe R, Reynolds S (eds.) \emph{African Paleoecology and Human Evolution}. 229-241. Cambridge University Press, Cambridge. 

\subsubsection*{2018}


\item {\bfseries Barr WA}. \emph{Ecomorphology}. In: Croft DA, Simpson SW, and Su DF (eds.), \emph{Methods in Paleoecology: Reconstructing Cenozoic Terrestrial Environments and Ecological Communities}. 339-349. Springer (Vertebrate Paleobiology and Paleoanthropology Series), Cham, Switzerland. \href{https://doi.org/10.1007/978-3-319-94265-0}{doi:10.1007/978-3-319-94265-0}

\item  Reed, DN, {\bfseries Barr WA}, Kappelman J. \emph{PaleoCore: an open-source platform for geospatial data integration in paleoanthropology}. In: Anemone R, Conroy G (eds.), \emph{New Geospatial Approaches in Anthropology}. University of New Mexico Press. Albuquerque, NM.
%\end{etaremune}




\section*{Courses Taught}

\begin{tabular}{p{.3\textwidth}p{0.75\textwidth}}
Cumulative undergrad enrollment & 1296 students = 5072 credit hours (updated Jan 24)\\[4pt]
Cumulative graduate enrollment &  185 students = 485 credit hours (updated Jan 24)
\end{tabular}


\begin{mylist}

\item[]\emph{Introduction to Biological Anthropology}. ANTH 1001. Undergraduate survey (enrollment=272) of biological anthropology. The George Washington University, Anthropology. Taught Fall 2016, Spring 2017, Spring 2018, Spring 2021, Spring 2023, Summer 2024.

\item[] \emph{Analytical Methods in Evolutionary Anthropology}. ANTH 6413. I designed this graduate course course covering applied statistical methods (e.g, regression, ANOVA and related techniques, categorical data analysis, resampling approaches) and the R statistical programming language. Requirement for HOMPAL PhD and MS degrees. The George Washington University, Anthropology. Taught Spring 2015, Fall 2016, Fall 2018, Fall 2020, Fall 2021, Fall 2022, Fall 2023. 

\item[] \emph{Hominid Paleobiology}. HOMP 6201. Graduate level survey of the human fossil record, with an emphasis on how we know what we think we know about human evolution. Requirement for HOMPAL PhD and MS degrees.    The George Washington University. Taught Fall 2019, Fall 2021. 

\item[] \emph{Climate Change and Human Evolution}. ANTH 3491. I designed this upper level undergraduate course covering changes in global climate through evolutionary time and the impacts on evolution, with an emphasis on humans. The George Washington University, Anthropology. Taught Spring 2016, Spring 2017, Fall 2022.

\item[] \emph{Hominin Evolution}. ANTH 3412 - advanced undergraduate course on the human fossil record. The George Washington University, Anthropology. Taught Spring 2021, Spring 2023.

\item[] \emph{Laboratory Techniques}. HOMP 6202. Survey of laboratory techniques in evolutionary anthropology. Requirement for HOMPAL PhD and MS degrees. The George Washington University. Taught Fall 2019, Fall 2020. 

\item[] \emph{Ethics and Professional Practice in Evolutionary Anthropology}. HOMP 6203. Survey of ethical considerations and professional development topics in evolutionary anthropology. Requirement for HOMPAL PhD and MS degrees. The George Washington University. Taught Fall 2023.

\item[] \emph{Public Understanding of Science}. HOMP 8302.  Graduate course in which students complete semester-long public service internships. Student projects target underserved Washington, DC-area public schools and general audiences at public museums with a goal of increasing scientific literacy and creating interest in scientific careers. The George Washington University, Anthropology. Taught Spring 2016.

%\item[] \emph{GIS and Remote Sensing for Archaeology and Paleontology}. ANT 391 / GRG 396. Teaching Assistant.  University of Texas at Austin, Anthropology. 2010.

%\item[] \emph{Human Variation}. ANT 394C. Teaching Assistant. University of Texas at Austin, Anthropology. 2009.

%\item[] \emph{Introduction to Physical Anthropology}. ANT 301. Teaching Assistant. University of Texas at Austin, Anthropology. 2006, 2007, 2010, 2011, 2012.
\end{mylist}



\section*{Honors and Awards - Excluding Student Prizes}

\begin{tabular}{p{.14\textwidth}p{0.82\textwidth}}
2024 & Nominated for GWU Morton A. Bender Teaching Award (not awarded)\\[4pt]
2023 & Nominated for GWU OVPR Research Mentorship Award (not awarded)\\[4pt]
2019 & Science Achievement Award - Smithsonian National Museum of Natural History - in recognition of outstanding research contributions for Tóth et al., 2019.\\[4pt]
2018 & American Association of Physical Anthropologists - Professional Development Award. \$7,500\\[4pt]
%2015 & Travel Grant - Paleoanthropology Society. \$500.\\[4pt]
%2013 & Pollitzer Student  Award - American Association of Physical Anthropologists. \$500.\\[4pt]
%2008 - 2011 & Professional Development Award - Department of Anthropology, University of Texas at Austin.\\[4pt]
%2007 & Student Prize - Texas Association of Biological Anthropologists.\\
\end{tabular}



\section*{Fieldwork }
\begin{longtable}{p{.16\textwidth}p{0.80\textwidth}}

2018 - Present & Tumbili Paleoanthropology Project, Laikipia County, Kenya (Late Miocene). I am the director of this new collaborative project in a unique geographical context outside the Great Rift Valley. To date we have discovered a moderately rich fossil fauna, and excavations are planned to expand the faunal sample from this poorly represented time period.\\[4pt]

2014 - Present & Mille-Logya Research Project, Afar Region, Ethiopia (Plio-Pleistocene). I conduct field research to recover new fossils and to understand the environmental and ecological context of human evolution in this region. \\[4pt]

2016 & Koobi-Fora Field School, East Turkana, Kenya. I collected fossil data relating to sub-regional faunal variability in the Koobi Fora Formation from 2.0 - 1.4 Ma. I supervised four undergraduate student research projects that were organized around this topic. \\[4pt]


2013 - 2014 & Great Divide Basin Project, Wyoming. Collected primate and mammalian fossils from Eocene sediments, and prospected for new localities. \\[4pt]

2010, 2012 & Dikika Research Project, Afar Region, Ethiopia. Surface collection of Plio- Pleistocene hominin and mammalian fossils. Managed GIS data collection with hand-held computers and high-precision GPS base station.\\[4pt]

2007, 2008, 2010 & Dalquest Research Site, Big Bend Region, Texas. Surface collected primate and mammalian fossils in the Devil's Graveyard Formation. (Eocene: Late Uintan).\\[4pt]

2009 & Contrebandiers Cave, Temara, Morocco.  Excavated site preserving Middle Stone Age archaeology (Aterian) and hominin remains. Performed systematic analysis of rodent fauna.\\
\end{longtable}

\section*{Synergistic Activities}
\begin{tabular}{p{.16\textwidth}p{0.8\textwidth}}

Summer 2021 & Participant. GW Instructional Core Course Design Institute. Five-day bootcamp during which I learned to implement best-practices of course design centered on authentic assessment, active learning, and consideration of student motivation.\\[4pt]

2015 - Present & External Member. Evolution of Terrestrial Ecosystems Working Group. National Museum of Natural History.\\[4pt]

2012 - Present & Research Associate and Software Developer. PaleoCore Project. I am a key member of this NSF funded project, which aims to create a data-standard for biological anthropology. I contributed heavily to the development of PaleoCore informatics tools for data sharing.\\
\end{tabular}

\section*{Scholarly Presentations}
\subsection*{Invited Talks, Symposia, Workshops}

\begin{etaremune}

\item 2023. Invited speaker. \emph{Hominin paleoenvironments, herbivore ecomorphology, and spatial bias in the African fossil record.} The University of Chicago. Evolutionary Morphology Seminar Series. 

\item 2023. Invited speaker.  \emph{Thinking outside the rift: some attempts to quantify the effect of spatial bias in the eastern African human fossil record}. Smithsonian National Museum of Natural History. Human Origins seminar series.  

\item 2019. Invited speaker. \emph{The Environmental Context of Hominin Evolution: Fieldwork, Fossils, and Functional Morphology}. Howard University, Washington, DC.

\item 2019. Invited speaker. \emph{The Environmental Context of Hominin Evolution: Fieldwork, Fossils, and Functional Morphology}. Colorado State University, Ft. Collins, Colorado.


\item 2018. Invited speaker at symposium: \emph{Advances in Paleoecology}. 2nd Lembersky Conference in Human Evolutionary Studies. Rutgers University. November 14 - 16.

\item 2017. Invited speaker. \emph{Data Analysis, Visualization, and Comparative Methods in R}. February 16-17. University of North Carolina - Greensboro.

\item 2015. Invited speaker at symposium: \emph{Latest methods in reconstructing Cenozoic terrestrial environments and ecological communities}. September 10 - 12. Cleveland Museum of Natural History.

\item 2014. Invited speaker at symposium: \emph{The Role of Mosaic Habitats in Hominin Evolution}. Annual Meeting of the American Association of Physical Anthropologists. Calgary, Alberta.
\end{etaremune}


\subsection*{Published Abstracts from Conference Presentations}

\begin{mylist}
\item[] *indicates undergraduate under my supervision
\end{mylist}

\begin{etaremune}

\item 2024. {\bfseries Barr WA}, Wood BA. Thinking beyond the rift: how important is sampling bias in understanding human evolution? Paleoanthropology Society. Los Angeles, CA. 

\item 2023. {\bfseries Barr WA}, Wood BA. Thinking outside the rift: Exploring the limitations of hominin habitat reconstructions based on spatially restricted species occurrences. European Society of Human Evolution.  Aarhus, Denmark. 

\item 2023. {\bfseries Barr WA}. Blinded by the rift? Exploring the limitations of hominin habitat reconstructions based on spatially restricted species occurrences. American Association of Biological Anthropologists.  Reno, Nevada.


\item 2022. Negash EW, {\bfseries Barr WA}. Landscape Level Vegetation Study in Modern African Ecosystems: Implications for Hominin Environments. Paleoanthropology Society. Denver, Colorado. 

 \item 2022. Robinson, JR, Rowan J, {\bfseries Barr WA}, Sponheimer M. Linking mammalian herbivore diets and (paleo)environments: implications for hominin paleoecology. American Association of Biological Anthropologists. Denver, Colorado. 


\item 2020. {\bfseries Barr WA}, Geraads D, Reed D, Bobe R, Wynn JG, Alemseged Z. Faunal turnover at Mille-Logya (Plio-Pleistocene, Ethiopia) reflects in situ environmental change: implications for the origins of \emph{Homo}. American Association of Physical Anthropologists. In-person meeting cancelled due to COVID-19.


\item 2019. Alemseged Z, Wynn JG, Geraads D, Reed D, {\bfseries Barr WA}, Bobe R, McPherron S. New hominin remains from Mille-Logya, Afar, Ethiopia and their implication for the origin of \emph{Homo}. Paleoanthropology Society.

\item 2018. Hammond AS, Hunter LE Thompson B, Corniner E, Biernat M, {\bfseries Barr WA}, Braun DR. Morphology and context of a new early \emph{Homo} mandible from Koobi Fora, Kenya.

\item 2018. {\bfseries Barr WA}, Biernat M*. Quantifying African habitat heterogeneity and mammalian functional diversity with implications for understanding hominin habitats. American Association of Physical Anthropology.


\item 2017. {\bfseries Barr WA}. Bovid locomotor traits track land cover and mean annual precipitation: using an ecometric approach to reconstruct paleoenvironments in the Shungura Formation (Plio-Pleistocene, Ethiopia). American Association of Physical Anthropology.

\item 2017,. Llera C*, Benitez L*, Biernat M*, Braun DR,  Hammond AS, Patterson DB, and {\bfseries Barr WA}. Subregion-scale heterogeneity in bovid abundance in the Koobi Fora Formation (Pleistocene, Northern Kenya).  American Association of Physical Anthropology.

\item 2017. Thompson B, Arenson J, Biernat M*,  {\bfseries Barr WA}, Reeves J, Braun DR and Hammond AS. A preliminary study of primate abundance in East Turkana collection areas relative to outcrop size. American Association of Physical Anthropology.

\item 2017. Enny A*, Biernat M*, Braun DR, Reda W*, Hammond AS, Patterson DB and {\bfseries Barr WA}. Exploring the impact of collection strategies on interpretations of faunal abundance: a case study from the Koobi Fora Formation (Pleistocene, northern Kenya). American Association of Physical Anthropology.

\item 2017. Benitez L*, Llera* C, Biernat M*, Braun DR, Hammond AS, Patterson DB, {\bfseries Barr WA}. The Implications of Faunal Abundance for Pleistocene Paleoenvironments in the Turkana Basin, Northern Kenya. Paleoanthropology Society. 


\item 2016. {\bfseries Barr WA}. Signal or noise? Testing hypotheses about faunal turnover. Paleoanthropology Society. 


\item 2015. {\bfseries Barr WA} and Dunn RH. A method for analyzing complex joint surfaces in ecomorphology using slope rasters derived from Digital Elevation Models. American Association of Physical Anthropology.
Thompson JC, McPherron SP, Bobe R, {\bfseries Barr WA}, Reed D, Wynn J, Marean CW, and Alemseged Z. Taphonomy of fossils from the hominin-bearing deposits at Dikika, Ethiopia. Paleoanthropology Society.


\item 2014. {\bfseries Barr WA}. Paleoenvironments of the Hadar and Shungura Formations: Synthesizing multiple lines of evidence using bovid ecomorphology. American Association of Physical Anthropology.

\item 2014. Kemp A and {\bfseries Barr WA}. Rates of homoplasy in the mammalian skeleton. American Association of Physical Anthropology.


\item 2013. {\bfseries Barr WA}. Ecomorphology of the bovid astragalus: body size, function, phylogeny and paleoenvironmental reconstruction. \emph{American Journal of Physical Anthropology}. 150:74.


\item 2012. {\bfseries Barr WA}. Ecomorphology in a phylogenetic statistical context: a case study using the bovid femur. \emph{American Journal of Physical Anthropology}. 147:90-91.

\item 2012. Scott RS and {\bfseries Barr WA}. Ecomorphology and phylogeny among the Bovidae: implications for habitat reconstruction. \emph{American Journal of Physical Anthropology}.

\item 2012. Kappelman JK, Keane P, Reed D, Tenbarge J, Witzel A, {\bfseries Barr WA}, Nachman BA, Russo GA. eFossils.org: a collaborative website and community database for the study of human evolution. \emph{American Journal of Physical Anthropology}.


\item 2011. Reed DN, McPherron S, {\bfseries Barr WA}, Alemseged Z, Bobe R, Geraads D, and Wynn J. A new GPS data collection methodology and data schema for integrating multiple project databases: examples from the Dikika Research Project geodatabase. \emph{American Journal of Physical Anthropology}. 144:249-250.


\item 2009. {\bfseries Barr WA}, Reed DN. Coping with taxonomic ambiguity and inter-observer variation in paleontological and paleoanthropological analyses. \emph{American Journal of Physical Anthropology}. 144:249-250.


\item 2009. Toborowsky CJ, {\bfseries Barr WA}, Lewis, RJ. Does environmental unpredictability drive lemur life histories? \emph{American Journal of Physical Anthropology}.


\item 2009. {\bfseries Barr WA}. The effects of allometric scaling patterns on the template method for estimating dimorphism. \emph{American Journal of Physical Anthropology}.


\end{etaremune}



\section*{Trainee Advising}
\begin{tabular}{p{.15\textwidth}p{.14\textwidth}p{0.70\textwidth}}
Postdoc & 2019-2022 & Enquye Negash (GWU)\\[4pt]
PhD (primary) & In progress & Kathyrn Fish (GWU)\\[4pt] %first semester F2020
PhD (primary) & In progress & Nick Rosas (GWU)\\[4pt] %first semester F2022
PhD (committee) & 2025 & Niguss Baraki (GWU)\\[4pt]
PhD (committee) & 2025 & Rachel Nelson (GWU)\\[4pt]
PhD (committee) & 2024 & Kristen Tuosto (GWU)\\[4pt]
PhD (committee) & 2024 & Alexis Williams (GWU)\\[4pt]
PhD (committee) & 2023 & Victoria Lockwood (GWU)\\[4pt]
PhD (committee) & 2023 & Ryan McRae (GWU)\\[4pt]
PhD (committee) & 2022 & Kim Foecke (GWU)\\[4pt]
PhD (committee) & 2019 & Eve Boyle (GWU)\\[4pt]
PhD (committee) & 2018 & Laurence Dumouchel (GWU)\\[4pt]
PhD (committee) & 2018 & Vance Powell (GWU)\\[4pt]
PhD (committee) & 2017 & Chrisandra Kufeldt (GWU)\\[4pt]
PhD (committee) & 2016 & David Patterson (GWU)\\

MS & In Progress & Isabel Coyne (GWU)\\[4pt]%first semester F2024
MS & In Progress & Anita Patane (GWU)\\[4pt]%first semester F2024
MS & In Progress & Kaitlin Sennewald (GWU)\\[4pt]%first semester F2024
MS & 2024 & Noelle Purcell (GWU)\\[4pt]%first semester F2020
MS & 2024 & Caitlyn Broderick (GWU)\\[4pt]%first semester F2022
%MS & In progress & Elissa Hurley (GWU)\\[4pt]%first semester F2023
MS & 2023 & Alyssa McGrath (GWU)\\[4pt]%first semester F2021
MS & 2021 & Annelise Beer (GWU)\\[4pt]
MS & 2021 & Monica Cheung (GWU)\\[4pt]
MS & 2020 & Nicholas Burns (GWU)\\[4pt] %first semester F2018
\end{tabular}



\subsection*{Undergraduate Research Associates}
\begin{tabular}{p{.14\textwidth}p{0.82\textwidth}}
2022 - present &  Sloan Fridrich (The George Washington University)\\[4pt]
2022 - present &  Emily LaBrasciano (The George Washington University)\\[4pt]
2022 - present &  Sophie Muir (The George Washington University)\\[4pt]
2022 - present & Olivia Poole (The George Washington University)\\[4pt]
2019 - 2020 & Rowan Sherwood (The George Washington University)\\[4pt]
2018 - 2019 & Jane Meiter (The George Washington University)\\[4pt]
2018 - 2019 & Paulette Ma (The George Washington University)\\[4pt]
2016 - 2017 & Maryse Biernat (Stockton University)\\[4pt]
2016 - 2017 & Elliot Greiner (The George Washington University)\\
\end{tabular} 

\subsection*{Koobi Fora Field School Students}
\begin{tabular}{p{.14\textwidth}p{0.82\textwidth}}
2018 & Suzy Strubel (University of Minnesota)\\[4pt]
2018 & Joshua Porter (The George Washington University)\\[4pt]
2018 & James Frazier (Bryn Mawr)\\[4pt] 
2018 & Annalys Hanson (Emory University)\\[4pt]
2016 & Lorena Benitez (Harvard University)\\[4pt]
2016 & Alyssa Enny (Stockton University)\\[4pt]
2016 & Catherine Llera (University of Florida)\\[4pt]
2016 & Weldeyared Reda (Aksum University, Ethiopia)\\
\end{tabular}
 



%\subsection*{Scholarly Presentations Without Published Abstracts}
%
%\begin{longtable}{p{.1\textwidth}p{0.86\textwidth}}
%2013 & {\bfseries Barr WA}, Nachman B, Shapiro L. The Academic Phylogeny of Physical Anthropology. Annual Meetings of the Texas Association for Biological Anthropologists. Austin, TX.\\[4pt]
%
%2013 & Reed D, {\bfseries Barr WA}, Urban T. Free as in speech and beer: open source software solutions for spatial data management in physical anthropology. Annual Meetings of the Texas Association for Biological Anthropologists. Austin, TX.\\[4pt]
%
%2012 & {\bfseries Barr WA}. Refining hominin paleoenvironmental reconstructions using bovid ecomorphology: the role of phylogenetic comparative methods. Presentation at University of Texas at Austin Paleontology Brown-Bag Seminar Series.\\[4pt]
%
%2012 & {\bfseries Barr WA}. Refining hominin paleoenvironmental reconstructions with bovid ecomorphology. Presentation at University of Texas at Austin Informal Physical Anthropology Semininar series.\\[4pt]
%
%2011 & {\bfseries Barr WA}. Ecomorphology in a phylogenetic statistical context: a case study using the bovid femur. Annual Meetings of Texas Association for Biological Anthropologists. San Marcos, TX.\\[4pt]
%
%2010 & {\bfseries Barr WA}. Pattern or chaos? Exploring a null model of faunal turnover patterns. Annual Meetings of Texas Association for Biological Anthropologists. Waco, TX.\\[4pt]
%
%2010 & {\bfseries Barr WA}. Quantitative ecomorphology of mammalian dentitions: Refining a tool for reconstructing early hominin paleoenvironments. Presentation at University of Texas at Austin Informal Physical Anthropology Semininar series.\\[4pt]
%
%2008 & {\bfseries Barr WA}. Coping with taxonomic ambiguity and inter-observer variation in paleontological and paleoanthropological analyses. Annual Meetings of Texas Association for Biological Anthropologists. College Station, TX.\\[4pt]
%
%2007 & {\bfseries Barr WA}. The effects of allometric scaling patterns on the template method for estimating dimorphism. Annual Meetings of Texas Association for Biological Anthropologists. Austin, TX.\\
%\end{longtable}




 
\section*{Departmental and Professional Service}
\begin{longtable}{p{.14\textwidth}p{0.82\textwidth}}
2020 - present & Director of Events - Center for the Advanced Study of Human Paleobiology.\\[4pt]
2023 - present & Diversity, Equity, and Inclusion Committee- GWU Anthropology\\[4pt]
2022 - present  & American Association of Biological Anthropologists - Cobb Professional Development Award Committee.\\[4pt]
2020 - present & CASHP Comprehensive Examination Committee - Member.\\[4pt]
2019 - 2020 & GWU Anthropology Library representative - Served as liason to GWU libraries on all aspects of library services on behalf of the Anthropology department.\\[4pt]
2017 - 2019 & Webmaster, Center for the Advanced Study of Human Paleobiology. I maintained the website and mailing list for our research center.\\[4pt]
2014 - 2015 & Coordinator, CASHP Journal Club - Coordinated speakers and organize the academic program for this departmental seminar in the Center for the Advanced Study of Human Paleobiology at The George Washington University. \\[4pt]
2013 - present &  \href{https://bioanthtree.org}{Academic Phylogeny of Biological Anthropology} - In collaboration with Liza Shapiro and Brett Nachman, I created this website as a public resource that tracks academic lineages of Biological Anthropology PhDs. The site has had over 2200 user submissions.\\
 %2008 & Reviewer - University of Texas Liberal Arts Graduate Research Fellowship. Evaluated grant proposals from students competing for \$50,000 in grant funds.\\
\end{longtable}

\section*{Public Outreach and Science Communication}
\begin{longtable}{p{.14\textwidth}p{0.82\textwidth}}
May 16, 2018 & \emph{The Scientist is In}. National Museum of Natural History. I interacted directly with over a hundred visitors to the National Museum of National History's Hall of Human Origins and answered their questions about human evolution.\\[4pt]
March 30, 2017 & \emph{Survivors: What Fossils Tell Us About the Past and Future}. I answered questions for the general public at the National Museum of National History as part of the 30th anniversary celebration of the Evolution of Terrestrial Ecosystems program, of which I am a member. \\[4pt]
2016 - Present & \href{https://twitter.com/facesfieldwork}{Faces of Fieldwork} - I created Faces of Fieldwork because I believe people engage more with science when they understand who we are and what we do. This twitter handle features photographic posts highlighting the good, the bad, and the ugly of real people doing real field research.\\[4pt]
%2014-2015 & Volunteer, Explore UT - University wide K-12 educational open house. Helped organize and run activity ``Leaping Lemurs of Madagascar'' on locomotion and conservation of lemurs.\\
\end{longtable}

\section*{Media Coverage}
\begin{longtable}{p{.14\textwidth}p{0.82\textwidth}}
2024 & Barr and Wood paper on spatial sampling bias in the fossil record covered by at least 20 news outlets. \href{https://www.altmetric.com/details/166500154/news}{altmetrics link}\\[4pt]
2022 & Barr et al. paper on human carnivory covered by over 60 news outlets including Wired, BBC, The Independent, NBC News, Science Magazine and Popular Science. The Smithsonian's flagship science podcast Sidedoor \href{https://www.si.edu/sidedoor/did-meat-make-us-human}{dedicated an entire episode to this paper} and \href{https://www.smithsonianmag.com/smithsonian-institution/fourteen-discoveries-made-about-human-evolution-in-2022-180981344/}{Smithsonian Magazine} ranked this paper as one of the top discoveries about human evolution in the year 2022. \href{https://www.altmetric.com/details/121513529/news}{altmetrics link}\\[4pt]
2019 & Tóth et al. paper on extinction impacts covered by 11 news outlets, including Smithsonian.com. \href{https://www.altmetric.com/details/66845562/news}{altmetrics link}\\[4pt]
2017 &  Online coverage of \emph{Signal or noise? A null model method for testing hypotheses about pulsed faunal turnover} in \href{https://www.sciencedaily.com/releases/2017/08/170804100410.htm}{Science Daily}, \href{https://phys.org/news/2017-08-paper-genus-homo-response-environmental.html}{phys.org}, \href{https://gwtoday.gwu.edu/origin-human-genus-may-have-occurred-chance}{GW Today} and others. \\[4pt]
2017 & Feature on \href{https://www.theguardian.com/lifeandstyle/2017/jul/01/pregnant-in-the-field-blog-photography-have-trowel-will-travel}{theguardian.com} highlighting several contributors to Faces of Fieldwork.\\[4pt]
2015 & \emph{GWU aims to be among top research schools.} \href{http://www.washingtonpost.com/local/education/gwu-aims-to-be-among-top-research-schools/2015/03/03/491da24e-c1f1-11e4-9ec2-b418f57a4a99_gallery.html}{Washington Post}.\\
\end{longtable}

\section*{Manuscript Reviews}
Science\\[4pt]
Proceedings of the National Academy of Sciences\\[4pt]
Journal of Human Evolution\\[4pt]
Nature Ecology \& Evolution\\[4pt]
Journal of Vertebrate Paleontology\\[4pt]
Methods in Ecology and Evolution\\[4pt]
Palaeogeography, Palaeoclimatology, Palaeoecology\\[4pt]
PLoS One\\[4pt]
Quaternary Research\\[4pt]
Comptes Rendus Palevol\\[4pt]
International Journal of Primatology\\[4pt]
Manning Publications (book proposal review)\\

\section*{Grant Reviews}
National Science Foundation\\[4pt]
The Leakey Foundation\\[4pt]
Deutsche Forschungsgemeinschaft (DFG) - German Research Foundation\\

\section*{Professional Memberships}
\begin{mylist}
\item[] American Association of Biological Anthropologists
\item[] Paleoanthropology Society
\item[] European Society for the Study of Human Evolution
\end{mylist}


\end{document}
